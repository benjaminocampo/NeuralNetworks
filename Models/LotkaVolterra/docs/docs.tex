%% bare_jrnl_compsoc.tex % V1.4b % 2015/08/26 % by Michael Shell % See: %
%http://www.michaelshell.org/ % for current contact information.
%%
%% This is a skeleton file demonstrating the use of IEEEtran.cls % (requires
%IEEEtran.cls version 1.8b or later) with an IEEE % Computer Society journal
%paper.
%%
%% Support sites: % http://www.michaelshell.org/tex/ieeetran/ %
%http://www.ctan.org/pkg/ieeetran % and % http://www.ieee.org/

%%*************************************************************************
%% Legal Notice: % This code is offered as-is without any warranty either
%expressed or % implied; without even the implied warranty of MERCHANTABILITY or
%% FITNESS FOR A PARTICULAR PURPOSE! % User assumes all risk. % In no event
%shall the IEEE or any contributor to this code be liable for % any damages or
%losses, including, but not limited to, incidental, % consequential, or any
%other damages, resulting from the use or misuse % of any information contained
%here.
%%
%% All comments are the opinions of their respective authors and are not %
%necessarily endorsed by the IEEE.
%%
%% This work is distributed under the LaTeX Project Public License (LPPL) %
%(http://www.latex-project.org/ ) version 1.3, and may be freely used, %
%distributed and modified. A copy of the LPPL, version 1.3, is included % in the
%base LaTeX documentation of all distributions of LaTeX released % 2003/12/01 or
%later. % Retain all contribution notices and credits. % ** Modified files
%should be clearly indicated as such, including  ** % ** renaming them and
%changing author support contact information. **
%%*************************************************************************


% *** Authors should verify (and, if needed, correct) their LaTeX system  ***
% *** with the testflow diagnostic prior to trusting their LaTeX platform ***
% *** with production work. The IEEE's font choices and paper sizes can   ***
% *** trigger bugs that do not appear when using other class files.       ***
% *** The testflow support page is at: http://www.michaelshell.org/tex/testflow/


\documentclass[10pt,journal,compsoc]{IEEEtran}
%
% If IEEEtran.cls has not been installed into the LaTeX system files, manually
% specify the path to it like:
% \documentclass[10pt,journal,compsoc]{../sty/IEEEtran}



% Some very useful LaTeX packages include: (uncomment the ones you want to load)


% *** MISC UTILITY PACKAGES ***
%
%\usepackage{ifpdf} Heiko Oberdiek's ifpdf.sty is very useful if you need
% conditional compilation based on whether the output is pdf or dvi. usage:
% \ifpdf % pdf code \else % dvi code \fi The latest version of ifpdf.sty can be
% obtained from: http://www.ctan.org/pkg/ifpdf Also, note that IEEEtran.cls V1.7
% and later provides a builtin \ifCLASSINFOpdf conditional that works the same
% way. When switching from latex to pdflatex and vice-versa, the compiler may
% have to be run twice to clear warning/error messages.


% *** CITATION PACKAGES ***
%
\ifCLASSOPTIONcompsoc
  % IEEE Computer Society needs nocompress option requires cite.sty v4.0 or
  % later (November 2003)
  \usepackage[nocompress]{cite}
\else
  % normal IEEE
  \usepackage{cite}
\fi
% cite.sty was written by Donald Arseneau V1.6 and later of IEEEtran pre-defines
% the format of the cite.sty package \cite{} output to follow that of the IEEE.
% Loading the cite package will result in citation numbers being automatically
% sorted and properly "compressed/ranged". e.g., [1], [9], [2], [7], [5], [6]
% without using cite.sty will become [1], [2], [5]--[7], [9] using cite.sty.
% cite.sty's \cite will automatically add leading space, if needed. Use
% cite.sty's noadjust option (cite.sty V3.8 and later) if you want to turn this
% off such as if a citation ever needs to be enclosed in parenthesis. cite.sty
% is already installed on most LaTeX systems. Be sure and use version 5.0
% (2009-03-20) and later if using hyperref.sty. The latest version can be
% obtained at: http://www.ctan.org/pkg/cite The documentation is contained in
% the cite.sty file itself.
%
% Note that some packages require special options to format as the Computer
% Society requires. In particular, Computer Society  papers do not use
% compressed citation ranges as is done in typical IEEE papers (e.g., [1]-[4]).
% Instead, they list every citation separately in order (e.g., [1], [2], [3],
% [4]). To get the latter we need to load the cite package with the nocompress
% option which is supported by cite.sty v4.0 and later. Note also the use of a
% CLASSOPTION conditional provided by IEEEtran.cls V1.7 and later.





% *** GRAPHICS RELATED PACKAGES ***
%
\ifCLASSINFOpdf \usepackage[pdftex]{graphicx}
  %declare the path(s) where your graphic files
  \graphicspath{{../pdf/}{../jpeg/}} % and their extensions so you won't
  % have to specify these with every instance of \includegraphics
  \DeclareGraphicsExtensions{.pdf,.jpeg,.png}
  \usepackage[justification=centering]{caption}
\else
  % or other class option (dvipsone, dvipdf, if not using dvips). graphicx will
  % default to the driver specified in the system graphics.cfg if no driver is
  % specified. \usepackage[dvips]{graphicx} declare the path(s) where your
  % graphic files are \graphicspath{{../eps/}} and their extensions so you won't
  % have to specify these with every instance of \includegraphics
  % \DeclareGraphicsExtensions{.eps}
\fi
% graphicx was written by David Carlisle and Sebastian Rahtz. It is required if
% you want graphics, photos, etc. graphicx.sty is already installed on most
% LaTeX systems. The latest version and documentation can be obtained at:
% http://www.ctan.org/pkg/graphicx Another good source of documentation is
% "Using Imported Graphics in LaTeX2e" by Keith Reckdahl which can be found at:
% http://www.ctan.org/pkg/epslatex
%
% latex, and pdflatex in dvi mode, support graphics in encapsulated postscript
% (.eps) format. pdflatex in pdf mode supports graphics in .pdf, .jpeg, .png and
% .mps (metapost) formats. Users should ensure that all non-photo figures use a
% vector format (.eps, .pdf, .mps) and not a bitmapped formats (.jpeg, .png).
% The IEEE frowns on bitmapped formats which can result in "jaggedy"/blurry
% rendering of lines and letters as well as large increases in file sizes.
%
% You can find documentation about the pdfTeX application at:
% http://www.tug.org/applications/pdftex






% *** MATH PACKAGES ***
%
\usepackage{amsmath}
\usepackage{cancel}
% that provides many useful and powerful commands for dealing with mathematics.
%
% Note that the amsmath package sets \interdisplaylinepenalty to 10000 thus
% preventing page breaks from occurring within multiline equations. Use:
% \interdisplaylinepenalty=2500 after loading amsmath to restore such page
% breaks as IEEEtran.cls normally does. amsmath.sty is already installed on most
% LaTeX systems. The latest version and documentation can be obtained at:
% http://www.ctan.org/pkg/amsmath





% *** SPECIALIZED LIST PACKAGES ***
%
%\usepackage{algorithmic} algorithmic.sty was written by Peter Williams and
% Rogerio Brito. This package provides an algorithmic environment fo describing
% algorithms. You can use the algorithmic environment in-text or within a figure
% environment to provide for a floating algorithm. Do NOT use the algorithm
% floating environment provided by algorithm.sty (by the same authors) or
% algorithm2e.sty (by Christophe Fiorio) as the IEEE does not use dedicated
% algorithm float types and packages that provide these will not provide correct
% IEEE style captions. The latest version and documentation of algorithmic.sty
% can be obtained at: http://www.ctan.org/pkg/algorithms Also of interest may be
% the (relatively newer and more customizable) algorithmicx.sty package by Szasz
% Janos: http://www.ctan.org/pkg/algorithmicx




% *** ALIGNMENT PACKAGES ***
%
%\usepackage{array} Frank Mittelbach's and David Carlisle's array.sty patches
% and improves the standard LaTeX2e array and tabular environments to provide
% better appearance and additional user controls. As the default LaTeX2e table
% generation code is lacking to the point of almost being broken with respect to
% the quality of the end results, all users are strongly advised to use an
% enhanced (at the very least that provided by array.sty) set of table tools.
% array.sty is already installed on most systems. The latest version and
% documentation can be obtained at: http://www.ctan.org/pkg/array


% IEEEtran contains the IEEEeqnarray family of commands that can be used to
% generate multiline equations as well as matrices, tables, etc., of high
% quality.




% *** SUBFIGURE PACKAGES *** \ifCLASSOPTIONcompsoc
%\usepackage[caption=false,font=footnotesize,labelfont=sf,textfont=sf]{subfig}
%\else \usepackage[caption=false,font=footnotesize]{subfig} \fi subfig.sty,
%written by Steven Douglas Cochran, is the modern replacement for subfigure.sty,
%the latter of which is no longer maintained and is incompatible with some LaTeX
%packages including fixltx2e. However, subfig.sty requires and automatically
%loads Axel Sommerfeldt's caption.sty which will override IEEEtran.cls' handling
%of captions and this will result in non-IEEE style figure/table captions. To
%prevent this problem, be sure and invoke subfig.sty's "caption=false" package
%option (available since subfig.sty version 1.3, 2005/06/28) as this is will
%preserve IEEEtran.cls handling of captions. Note that the Computer Society
%format requires a sans serif font rather than the serif font used in
%traditional IEEE formatting and thus the need to invoke different subfig.sty
%package options depending on whether compsoc mode has been enabled.
%
% The latest version and documentation of subfig.sty can be obtained at:
% http://www.ctan.org/pkg/subfig




% *** FLOAT PACKAGES ***
%
%\usepackage{fixltx2e} fixltx2e, the successor to the earlier fix2col.sty, was
% written by Frank Mittelbach and David Carlisle. This package corrects a few
% problems in the LaTeX2e kernel, the most notable of which is that in current
% LaTeX2e releases, the ordering of single and double column floats is not
% guaranteed to be preserved. Thus, an unpatched LaTeX2e can allow a single
% column figure to be placed prior to an earlier double column figure. Be aware
% that LaTeX2e kernels dated 2015 and later have fixltx2e.sty's corrections
% already built into the system in which case a warning will be issued if an
% attempt is made to load fixltx2e.sty as it is no longer needed. The latest
% version and documentation can be found at: http://www.ctan.org/pkg/fixltx2e


%\usepackage{stfloats} stfloats.sty was written by Sigitas Tolusis. This package
% gives LaTeX2e the ability to do double column floats at the bottom of the page
% as well as the top. (e.g., "\begin{figure*}[!b]" is not normally possible in
% LaTeX2e). It also provides a command: \fnbelowfloat to enable the placement of
% footnotes below bottom floats (the standard LaTeX2e kernel puts them above
% bottom floats). This is an invasive package which rewrites many portions of
% the LaTeX2e float routines. It may not work with other packages that modify
% the LaTeX2e float routines. The latest version and documentation can be
% obtained at: http://www.ctan.org/pkg/stfloats Do not use the stfloats
% baselinefloat ability as the IEEE does not allow \baselineskip to stretch.
% Authors submitting work to the IEEE should note that the IEEE rarely uses
% double column equations and that authors should try to avoid such use. Do not
% be tempted to use the cuted.sty or midfloat.sty packages (also by Sigitas
% Tolusis) as the IEEE does not format its papers in such ways. Do not attempt
% to use stfloats with fixltx2e as they are incompatible. Instead, use Morten
% Hogholm'a dblfloatfix which combines the features of both fixltx2e and
% stfloats:
%
% \usepackage{dblfloatfix} The latest version can be found at:
% http://www.ctan.org/pkg/dblfloatfix




%\ifCLASSOPTIONcaptionsoff \usepackage[nomarkers]{endfloat}
%  \let\MYoriglatexcaption\caption
%  \renewcommand{\caption}[2][\relax]{\MYoriglatexcaption[#2]{#2}} \fi
%  endfloat.sty was written by James Darrell McCauley, Jeff Goldberg and Axel
%  Sommerfeldt. This package may be useful when used in conjunction with
%  IEEEtran.cls'  captionsoff option. Some IEEE journals/societies require that
%  submissions have lists of figures/tables at the end of the paper and that
%  figures/tables without any captions are placed on a page by themselves at the
%  end of the document. If needed, the draftcls IEEEtran class option or
%  \CLASSINPUTbaselinestretch interface can be used to increase the line spacing
%  as well. Be sure and use the nomarkers option of endfloat to prevent endfloat
%  from "marking" where the figures would have been placed in the text. The two
%  hack lines of code above are a slight modification of that suggested by in
%  the endfloat docs (section 8.4.1) to ensure that the full captions always
%  appear in the list of figures/tables - even if the user used the short
%  optional argument of \caption[]{}. IEEE papers do not typically make use of
%  \caption[]'s optional argument, so this should not be an issue. A similar
%  trick can be used to disable captions of packages such as subfig.sty that
%  lack options to turn off the subcaptions: For subfig.sty:
%  \let\MYorigsubfloat\subfloat
%  \renewcommand{\subfloat}[2][\relax]{\MYorigsubfloat[]{#2}} However, the above
%  trick will not work if both optional arguments of the \subfloat command are
%  used. Furthermore, there needs to be a description of each subfigure
%  *somewhere* and endfloat does not add subfigure captions to its list of
%  figures. Thus, the best approach is to avoid the use of subfigure captions
%  (many IEEE journals avoid them anyway) and instead reference/explain all the
%  subfigures within the main caption. The latest version of endfloat.sty and
%  its documentation can obtained at: http://www.ctan.org/pkg/endfloat
%
% The IEEEtran \ifCLASSOPTIONcaptionsoff conditional can also be used later in
% the document, say, to conditionally put the References on a page by
% themselves.




% *** PDF, URL AND HYPERLINK PACKAGES ***
%
%\usepackage{url} url.sty was written by Donald Arseneau. It provides better
% support for handling and breaking URLs. url.sty is already installed on most
% LaTeX systems. The latest version and documentation can be obtained at:
% http://www.ctan.org/pkg/url Basically, \url{my_url_here}.





% *** Do not adjust lengths that control margins, column widths, etc. *** *** Do
% not use packages that alter fonts (such as pslatex).         *** There should
% be no need to do such things with IEEEtran.cls V1.6 and later. (Unless
% specifically asked to do so by the journal or conference you plan to submit
% to, of course. )


% correct bad hyphenation here
\hyphenation{op-tical net-works semi-conduc-tor}


\begin{document}
%
% paper title Titles are generally capitalized except for words such as a, an,
% and, as, at, but, by, for, in, nor, of, on, or, the, to and up, which are
% usually not capitalized unless they are the first or last word of the title.
% Linebreaks \\ can be used within to get better formatting as desired. Do not
% put math or special symbols in the title.
\title{Modelo de Predadores y Prezas de Lotka-Volterra}
%
%
% author names and IEEE memberships note positions of commas and nonbreaking
% spaces ( ~ ) LaTeX will not break a structure at a ~ so this keeps an author's
% name from being broken across two lines. use \thanks{} to gain access to the
% first footnote area a separate \thanks must be used for each paragraph as
% LaTeX2e's \thanks was not built to handle multiple paragraphs
%
%
%\IEEEcompsocitemizethanks is a special \thanks that produces the bulleted lists
% the Computer Society journals use for "first footnote" author affiliations.
% Use \IEEEcompsocthanksitem which works much like \item for each affiliation
% group. When not in compsoc mode, \IEEEcompsocitemizethanks becomes like
% \thanks and \IEEEcompsocthanksitem becomes a line break with idention. This
% facilitates dual compilation, although admittedly the differences in the
% desired content of \author between the different types of papers makes a
% one-size-fits-all approach a daunting prospect. For instance, compsoc journal
% papers have the author affiliations above the "Manuscript received ..."  text
% while in non-compsoc journals this is reversed. Sigh.

\author{\textbf{Nicolás Benjamín Ocampo}\\
\textit{Licenciatura en Ciencias de la Computación - FaMAF}\\
\textit{Redes Neuronales}
}

% note the % following the last \IEEEmembership and also \thanks - these prevent
% an unwanted space from occurring between the last author name and the end of
% the author line. i.e., if you had this:
%
% \author{....lastname \thanks{...} \thanks{...} }
%                     ^------------^------------^----Do not want these spaces!
%
% a space would be appended to the last name and could cause every name on that
% line to be shifted left slightly. This is one of those "LaTeX things". For
% instance, "\textbf{A} \textbf{B}" will typeset as "A B" not "AB". To get "AB"
% then you have to do: "\textbf{A}\textbf{B}" \thanks is no different in this
% regard, so shield the last } of each \thanks that ends a line with a % and do
% not let a space in before the next \thanks. Spaces after \IEEEmembership other
% than the last one are OK (and needed) as you are supposed to have spaces
% between the names. For what it is worth, this is a minor point as most people
% would not even notice if the said evil space somehow managed to creep in.



% The paper headers \markboth{Journal of \LaTeX\ Class Files,~Vol.~14, No.~8,
%August~2015}% {Shell \MakeLowercase{\textit{et al.}}: Bare Demo of IEEEtran.cls
%for Computer Society Journals} The only time the second header will appear is
%for the odd numbered pages after the title page when using the twoside option.
%
% *** Note that you probably will NOT want to include the author's *** *** name
% in the headers of peer review papers.                   *** You can use
% \ifCLASSOPTIONpeerreview for conditional compilation here if you desire.



% The publisher's ID mark at the bottom of the page is less important with
% Computer Society journal papers as those publications place the marks outside
% of the main text columns and, therefore, unlike regular IEEE journals, the
% available text space is not reduced by their presence. If you want to put a
% publisher's ID mark on the page you can do it like this:
% \IEEEpubid{0000--0000/00\$00.00~\copyright~2015 IEEE} or like this to get the
% Computer Society new two part style. \IEEEpubid{\makebox[\columnwidth]{\hfill
% 0000--0000/00/\$00.00~\copyright~2015 IEEE}%
% \hspace{\columnsep}\makebox[\columnwidth]{Published by the IEEE Computer
% Society\hfill}} Remember, if you use this you must call \IEEEpubidadjcol in
% the second column for its text to clear the IEEEpubid mark (Computer Society
% jorunal papers don't need this extra clearance.)



% use for special paper notices \IEEEspecialpapernotice{(Invited Paper)}



% for Computer Society papers, we must declare the abstract and index terms
% PRIOR to the title within the \IEEEtitleabstractindextext IEEEtran command as
% these need to go into the title area created by \maketitle. As a general rule,
% do not put math, special symbols or citations in the abstract or keywords.
\IEEEtitleabstractindextext{%
\begin{abstract}
Este articulo aborda la demografía de una población de conejos y de zorros que
conviven en un ecosistema determinado, siendo modelado a partir de las
ecuaciónes \textit{predador-presa} de \textbf{Lotka-Volterra}. En particular,
haremos uso de sistemas dinámicos con el fin de comprender el cambio poblacional
de ambas especies cuando se deja evolucionar el sistema libremente. Realizaremos
un análisis general de la estabilidad del mismo para luego concluir sobre un
caso en particular bajo ciertas condiciones iniciales y parametros influyentes
dentro de nuestro modelo. Finalmente abordaremos la pregunta si los predadores y
prezas pueden convivir entre ellos.
\end{abstract}

% Note that keywords are not normally used for peerreview papers.
\begin{IEEEkeywords}
Lotka-Volterra, Predadores, Prezas, Sistemas Dinámicos \end{IEEEkeywords}}


% make the title area
\maketitle


% To allow for easy dual compilation without having to reenter the
% abstract/keywords data, the \IEEEtitleabstractindextext text will not be used
% in maketitle, but will appear (i.e., to be "transported") here as
% \IEEEdisplaynontitleabstractindextext when the compsoc or transmag modes are
% not selected <OR> if conference mode is selected 
% - because all conference papers position the abstract like regular papers do.
\IEEEdisplaynontitleabstractindextext
% \IEEEdisplaynontitleabstractindextext has no effect when using compsoc or
% transmag under a non-conference mode.



% For peer review papers, you can put extra information on the cover page as
% needed: \ifCLASSOPTIONpeerreview \begin{center} \bfseries EDICS Category:
% 3-BBND \end{center} \fi
%
% For peerreview papers, this IEEEtran command inserts a page break and creates
% the second title. It will be ignored for other modes.
\IEEEpeerreviewmaketitle



\IEEEraisesectionheading{\section{Introducción}\label{sec:introduction}}
% Computer Society journal (but not conference!) papers do something unusual
% with the very first section heading (almost always called "Introduction").
% They place it ABOVE the main text! IEEEtran.cls does not automatically do this
% for you, but you can achieve this effect with the provided
% \IEEEraisesectionheading{} command. Note the need to keep any \label that is
% to refer to the section immediately after \section in the above as
% \IEEEraisesectionheading puts \section within a raised box.




% The very first letter is a 2 line initial drop letter followed by the rest of
% the first word in caps (small caps for compsoc).
%
% form to use if the first word consists of a single letter:
% \IEEEPARstart{A}{demo} file is ....
%
% form to use if you need the single drop letter followed by normal text
% (unknown if ever used by the IEEE): \IEEEPARstart{A}{}demo file is ....
%
% Some journals put the first two words in caps: \IEEEPARstart{T}{his demo} file
% is ....
%
% Here we have the typical use of a "T" for an initial drop letter and "HIS" in
% caps to complete the first word.
\subsection{Descripción del problema}
\IEEEPARstart{C}{}onsidera la siguiente situación sobre un ecosistema donde
 conviven tantos \textbf{zorros} como \textbf{conejos}. Ambas poblaciones pueden
 crecer o decrecer en cantidad a partir de un tiempo inicial $t = 0$ con una
 cierta \textbf{taza de crecimiento}. Ahora bien, dado que los zorros son
 \textbf{predadores} de los conejos, necesitan de su existencia e interacción
 para sobrevivir. Por otro lado, dicha interacción, si bien beneficia a los
 zorros, afecta a los conejos al ser su \textbf{presa}. Estos intercambios están
 regulados por \textbf{tazas de éxito en la caza} en favor de los zorros y en
 contra los conejos.

 Las tazas de crecimiento y de éxito en la caza, están dadas por las
 caracteristicas de nuestro ecosistema.

 Dado este escenario, uno se podría preguntar:
 \begin{itemize}
   \item ¿Qué ocurre con la población de ambas especies cuando se deja
   evolucionar el sistema libremente, es decir, cuando $t$ tiende a infinito?
   \item ¿Pueden los conejos y los zorros coexistir?
   \item ¿Podrían extinguirse ambas especies? ¿Y alguna de ellas?
 \end{itemize}

 A priori no es fácil dar una respuesta a estas cuestiones, por ende
 realizaremos suposiciones que conciernen al dominio del problema y modelaremos
 la situación con el fin de analizarla utilizando las \textbf{ecuaciónes de
 predadores y prezas de Lotka-Volterra}.

 \subsection{Modelado del problema}
 El modelo Lotka-Volterra está determinado por las siguientes ecuaciones
 diferenciales.
 
 \begin{equation*}
   \begin{aligned}
    \dot C &= \alpha C - \beta CZ \\
    \dot Z &= -\gamma Z + \delta CZ
   \end{aligned}
 \end{equation*}

 \begin{itemize}
   \item $C$ : Cantidad de conejos en el ecosistema.
   \item $\alpha$ : Taza de crecimiento de los conejos.
   \item $\beta$ : Taza de éxito en la caza que afecta a la presa.
   \item $Z$ : Cantidad de zorros en el ecosistema.
   \item $\gamma$ : Taza de crecimiento de los zorros.
   \item $\delta$ : Taza de éxito en la caza que favorece al depredador.
 \end{itemize}

 Analicemos como estas ecuaciones describen nuestra situación.
 
 Para empezar, las expresiones $C$ y $Z$ son funciones incognitas que indican la
 cantidad poblacional en el tiempo, es decir, dependen de este ultimo ($C(t)$ y
 $Z(t)$).

 Por otro lado, $\dot C$ y $\dot Z$ son las \textbf{razónes de cambio} de dichas
 funciones en el tiempo. Por lo tanto nos va a interesar obtener información de
 $C(t)$ y $Z(t)$ a partir de sus tazas de cambio.

 Los parametros $\alpha, \beta, \gamma, \delta$ son constantes positivas que,
 como mencionamos, se obtienen a partir de caracteristicas de nuestro sistema.

 Notar que la razón de cambio de la población de los conejos crece con taza
 $\alpha$ y es afectado por la interacción con los zorros con taza de éxito
 $\beta$. Esto nos dice además que el alimento de los conejos en cierta manera
 es ilimitado influyente por la constante $\alpha$. Analogamente, la razón de
 cambio de la población de los zorros decrece con taza $\gamma$ y es favorecido
 por la interacción con los conejos con taza de éxito $\delta$. Dicho de otra
 manera el alimento de los zorros depende de la cantidad de presas que hayan en
 el entorno, siendo este último regulado por los parámetros.

 De allí se puede intuir que si no hubiesen zorros, la población de los conejos
 crecería indefinidamente. Por otro lado, si no hubiesen conejos, los zorros se
 extinguirían. Además, puede presenciarse un ciclo de crecimiento y
 declinamiento entre ambas especies, los predadores aumentan mientras las presas
 abunden, y las presas abundan siempre y cuando los predadores sean pocos.
 
 De todas formas plantearemos un análisis más riguroso que nos permita
 justificar nuestra intuición y nos deje tranquilos al final del día.

 Finalmente, concluiremos con un ejemplo de análisis para los siguientes
 parametros.
 \begin{equation*}
  \begin{aligned}
   \alpha &= 0.1\\
   \beta &= 0.02\\
   \gamma &= 0.3\\
   \delta &= 0.01
  \end{aligned}
 \end{equation*}

 Para este caso, la interacción entre presas y predadores afectara ligeramente
 más a los conejos que lo que favorecerá a los zorros. Notar también que la
 población de los zorros decrece con taza $\gamma = 0.3$ si este no consigue
 alimentos.
 \section{Diagrama de Flujo y Estabilidad del Sistema}

 \subsection{Puntos Fijos}
 Para determinar que va a ocurrir con nuestro sistema cuando el tiempo tienda a
 infinito, debemos encontrar primero los puntos fijos del mismo para luego
 analizar su estabilidad (Si nuestras funciones incognitas $C$ y $Z$ se apartan,
 atraen, u oscilan en dichos puntos). Veamos para que valores de $C$ y $Z$
 tenemos que $\dot C = \dot Z = 0$.
 
 Esta claro que un punto fijo del sistema es $(C^{*}, Z^{*})$ = $(0, 0)$.
 Además, notar que
 \begin{equation*}
   \begin{aligned}
    \dot C &= 0 \\
    \alpha C - \beta CZ &= 0\\
    \alpha \cancel{C} &= \beta \cancel{C}Z \\
    \alpha &= \beta Z \\
    \frac{\alpha}{\beta} &= Z \\
   \end{aligned}
 \end{equation*}
 
 Por otro lado,

 \begin{equation*}
  \begin{aligned}
   \dot Z &= 0 \\
   -\gamma Z + \delta CZ &= 0\\
   \delta C\cancel{Z} &= \gamma \cancel{Z} \\
   \delta C &= \gamma \\
   C &= \frac{\gamma}{\delta}
  \end{aligned}
 \end{equation*}

 Por lo tanto el otro punto fijo es $(C^{*}, Z^{*}) = (\frac{\gamma}{\delta},
 \frac{\alpha}{\beta})$

 \subsection{Linealización alrededor de los puntos fijos}

 La estabilidad de los puntos fijos puede ser determinada linealizando alrededor
 de ellos. Es decir, posicionarnos cerca de los puntos fijos por medio de una
 pequeña perturbación, desplazar nuestros ejes coordenados en dichos puntos y
 considerar nuevos ejes sobre los cuales la taza de cambio $\dot C$ no depende
 de $\dot Z$ y viceversa. Esto nos indicará que nuestro sistema es
 \textit{linealmente separable}.

 Para ello obtengamos la mátriz Jacobiana de nuestro modelo dada como

 \begin{equation*}
  \begin{aligned}
  A &= 
    \begin{pmatrix}
      \dfrac{\partial \dot C}{\partial C} & \dfrac{\partial \dot C}{\partial Z}\\
      & \\
      \dfrac{\partial \dot Z}{\partial C} & \dfrac{\partial \dot Z}{\partial Z} \\
    \end{pmatrix} \\
    &= \begin{pmatrix}
      \alpha - \beta Z & -\beta C\\
      & \\
      \delta Z & \delta C - \gamma
    \end{pmatrix} \\
  \end{aligned}
 \end{equation*}
 
 Luego evaluamos los puntos fijos $(C^{*}, Z^{*})$ en la matriz jacobiana.

 \subsubsection{Estabilidad del Punto Fijo (0, 0)}

 Para este caso la matriz nos queda
 
 \begin{equation*}
  \begin{aligned}
  A &= \begin{pmatrix}
      \alpha - \beta 0 & -\beta 0\\
      & \\
      \delta 0 & \delta 0 - \gamma
    \end{pmatrix} \\
    &= \begin{pmatrix}
      \alpha & 0 \\
      & \\
      0 & -\gamma
    \end{pmatrix} \\
  \end{aligned}
 \end{equation*}

 Luego para conocer la estabilidad de dicho punto fijo (Cercano a este punto
 fijo), necesitamos los autovalores de dicha matriz.
 
 Como $A$ es diagonal sus autovalores son $\lambda_1 = \alpha$ y $\lambda_2 =
 -\gamma$ cuya base de autovectores es la canonica $\vec{v_1} = (1, 0)$ y
 $\vec{v_2} = (0, 1)$.

 Esto nos dice que, cercano al punto fijo, no hay necesidad de un cambio de ejes
 coordenados pues nuestro sistema está desacoplado.
 
 Dado también que $\lambda_1$ y $\lambda_2$ nos resultan con signos opuestos
 (Recordar que los parámetros son positivos), tenemos que $(0, 0)$ es un
 \textbf{punto silla inestable}.

 Es decir, las trajectorias son atraidas al punto fijo bajo el eje delimitado
 por $\vec{v_2}$, sin embargo estas se repelen de manera exponencial bajo el eje
 delimitado por $\vec{v_1}$

 \subsubsection{Estabilidad del Punto Fijo $(\frac{\gamma}{\delta},
 \frac{\alpha}{\beta})$}
 
 Siguiendo el mismo procedimiento tenemos que

 \begin{equation*}
  \begin{aligned}
  A &= \begin{pmatrix}
      \alpha - \cancel{\beta} \frac{\alpha}{\cancel{\beta}} & -\beta \frac{\gamma}{\delta}\\
      & \\
      \delta \frac{\alpha}{\beta} & \cancel{\delta} \frac{\gamma}{\cancel{\delta}} - \gamma
    \end{pmatrix} \\
    &= \begin{pmatrix}
      \alpha - \alpha & -\beta \frac{\gamma}{\delta}\\
      & \\
      \delta \frac{\alpha}{\beta} &  \gamma - \gamma
    \end{pmatrix} \\
    &= \begin{pmatrix}
      0 & -\beta \frac{\gamma}{\delta}\\
      & \\
      \delta \frac{\alpha}{\beta} & 0
    \end{pmatrix}
  \end{aligned}
 \end{equation*}
 
 Luego para calcular los autovalores de dicha matriz debemos encontrar las
 raices de su polinomio caracteristico dado por:

 \begin{equation*}
  \begin{aligned}
  det \begin{pmatrix}
      -\lambda & -\beta \frac{\gamma}{\delta}\\
      & \\
      \delta \frac{\alpha}{\beta} & -\lambda
    \end{pmatrix}
    &= \lambda^{2} + \cancel{\delta} \frac{\alpha}{\cancel{\beta}}\cancel{\beta}\frac{\gamma}{\cancel{\delta}}
    &= \lambda^{2} + \alpha \gamma
  \end{aligned}
 \end{equation*}

 Luego los autovalores son $\lambda_{1,2} = \pm i\sqrt{\alpha \gamma}$.
 
 Dado que resultaron ser imaginarios puros, es decir con parte real nula,
 tenemos un \textbf{punto fijo centro}, que no es atractor, ni repelente, donde
 las trayectorias oscilan con forma de un centro alrededor de este punto.

 \subsection{Diagrama de Flujo}

 Con la información obtenida, estamos en condiciones de responder las preguntas
 que nos planteamos al inicio del articulo. Si bien no conocemos la solución
 exacta de nuestras incógnitas $C(t)$ y $Z(t)$, conocemos cuales son sus
 comportamiento a medida que el tiempo crece cuando estas están cerca de los
 puntos fijos. Luego de acuerdo a [1], está aproximación puede extrapolarse
 cuando se está contemplando valores mucho más apartados.
 
 Ahora bien, dicho comportamiento depende de cual fue la cantidad de zorros y de
 conejos inicial en nuestro ecosistema. Por lo tanto, para visualizar de manera
 más sencilla el proceder de $C(t)$ y $Z(t)$ confeccionaremos un
 \textbf{Diagrama de Flujo} del sistema que nos dará de manera cualitativa las
 posibles trayectorias de estas funciones bajo multiples condiciones iniciales.

 Como dijimos en la introducción, realizaremos nuestro análisis para la
 parametrización:

 \begin{equation*}
  \begin{aligned}
   \alpha &= 0.1\\
   \beta &= 0.02\\
   \gamma &= 0.3\\
   \delta &= 0.01
  \end{aligned}
 \end{equation*}

 Para este caso, los puntos fijos son $(0, 0)$ y $(\frac{\gamma}{\delta},
 \frac{\alpha}{\beta}) = (30, 5)$. Recordemos entonces que para
 $(0,0)$ obtuvimos un punto de ensilladura inestable, y para $(30, 5)$ un punto centro
 neutro.

 Como puede verse en la Figura 1 esto concuerda con nuestros resultados
 generales. Recordar también que para el punto $(0, 0)$ habiamos visto que sus autovalores
 eran $\vec{v_1} = (1, 0)$ y $\vec{v_2} = (0, 1)$ cuyas direcciones corresponden
 con las de los ejes $C$ y $Z$ respectivamente.
 
 \begin{figure}[!t]
  \centering
  \includegraphics[width=3in]{../Imagenes/DiagramaDeFlujo.png}
  \caption{Diagrama de Flujo del Sistema}
  \label{fig_sim}
 \end{figure}

 Además, el autovalor $\lambda_1$ es positivo por lo tanto las trayectorias
 tienden de manera exponencial sobre el eje $C$ en dirección positiva hacia la
 derecha apartandose del punto fijo. Para las trayectorias posicionadas en su
 eje contraparte $Z$, estas terminan siendo atraidas.

 También se puede ver que ocurre un comportamiento dual cuando $Z$ o $C$ son
 negativos, pero estos casos no serán de nuestro interés bajo un análisis
 biológico (Dado que no puede haber menor a 0 seres vivos de una especie).

 Notar que por medio del gráfico uno puede percatarse de los
 \textbf{nullclines}, es decir, aquellas regiones sobre las cuales el flujo es
 puramente horizontal o vertical. Estos casos se dan para aquellos flujos que
 rigen sobre los mismos ejes, es decir, cuando inicialmente no hay zorros o no
 hay conejos.

 Esto confirma la intuición que tuvimos al comienzo, si inicialmente no hay
 zorros, los conejos perdurarán infinitamente en el tiempo. Por otro lado si no
 hubiese conejos, los zorros no podrían sobrevivir.

 Finalmente, para el punto $(30, 5)$ se confirma nuevamente lo que obtuvimos en
 la sección anterior. Esto nos quiere decir que, si hubiesen zorros y conejos
 inicialmente, estos convivirían de manera indefinida en el tiempo donde en
 ocasiones prosperarían uno u otro. Por lo tanto ninguno de los 2 se
 extinguiría!

 \section{Aproximación Numérica}

 Abordemos ahora la situación desde otra perspectiva. Supongamos que partimos
 del tiempo inicial $t = 0$ con cantidades $Z_0 = 9$ de zorros y $C_0 = 40$ de
 conejos. Aproximemos las soluciones $C(t)$ y $Z(t)$ por medio de \textbf{Runge
 Kutta de Orden 4} y ver si obtenemos una trayectoria oscilante hasta un tiempo
 $t = 200$.

 \begin{figure}[!t]
  \centering
  \includegraphics[width=3in]{../Imagenes/CvsZ.png}
  \caption{\textbf{Z vs C} - Aproximación numérica al problema del valor inicial
  con $t = 0$, $Z_0 = 9$ y $C_0 = 40$}
  \label{fig_sim}
 \end{figure}

 \begin{figure}[!t]
  \centering
  \includegraphics[width=3in]{../Imagenes/CZvsT.png}
  \caption{\textbf{Z y C vs t} - Aproximación numérica al problema del valor
  inicial con $t = 0$, $Z_0 = 9$ y $C_0 = 40$}
  \label{fig_sim}
 \end{figure}

 Como puede verse en las Figuras 2 y 3, los resultados por medio de este método
 no contradicen a lo obtenido bajo el análisis de estabilidad. Se tiene que las
 soluciones estimadas de $C(t)$ y $Z(t)$ siguen un comportamiento oscilante y
 uniforme al transcurrir el tiempo, manteniendo una población alrededor de 15 a
 50 conejos y 2 a 13 zorros en nuestro ecosistema.

 Por lo tanto ambás formas de lidiar con el problema son totalmente válidas. Sin
 embargo, el segundo método nos deja con la intriga si para otras condiciones
 iniciales tendríamos un comportamiento similar.
% An example of a floating figure using the graphicx package. Note that \label
% must occur AFTER (or within) \caption. For figures, \caption should occur
% after the \includegraphics. Note that IEEEtran v1.7 and later has special
% internal code that is designed to preserve the operation of \label within
% \caption even when the captionsoff option is in effect. However, because of
% issues like this, it may be the safest practice to put all your \label just
% after \caption rather than within \caption{}.
%
% Reminder: the "draftcls" or "draftclsnofoot", not "draft", class option should
% be used if it is desired that the figures are to be displayed while in draft
% mode.
%
%\begin{figure}[!t] \centering \includegraphics[width=2.5in]{myfigure} where an
%.eps filename suffix will be assumed under latex, and a .pdf suffix will be
%assumed for pdflatex; or what has been declared via \DeclareGraphicsExtensions.
%\caption{Simulation results for the network.} \label{fig_sim} \end{figure}

% Note that the IEEE typically puts floats only at the top, even when this
% results in a large percentage of a column being occupied by floats. However,
% the Computer Society has been known to put floats at the bottom.


% An example of a double column floating figure using two subfigures. (The
% subfig.sty package must be loaded for this to work.) The subfigure \label
% commands are set within each subfloat command, and the \label for the overall
% figure must come after \caption. \hfil is used as a separator to get equal
% spacing. Watch out that the combined width of all the subfigures on a line do
% not exceed the text width or a line break will occur.
%
%\begin{figure*}[!t] \centering \subfloat[Case
%I]{\includegraphics[width=2.5in]{box}% \label{fig_first_case}} \hfil
%\subfloat[Case II]{\includegraphics[width=2.5in]{box}% \label{fig_second_case}}
%\caption{Simulation results for the network.} \label{fig_sim} \end{figure*}
%
% Note that often IEEE papers with subfigures do not employ subfigure captions
% (using the optional argument to \subfloat[]), but instead will
% reference/describe all of them (a), (b), etc., within the main caption. Be
% aware that for subfig.sty to generate the (a), (b), etc., subfigure labels,
% the optional argument to \subfloat must be present. If a subcaption is not
% desired, just leave its contents blank, e.g., \subfloat[].


% An example of a floating table. Note that, for IEEE style tables, the \caption
% command should come BEFORE the table and, given that table captions serve much
% like titles, are usually capitalized except for words such as a, an, and, as,
% at, but, by, for, in, nor, of, on, or, the, to and up, which are usually not
% capitalized unless they are the first or last word of the caption. Table text
% will default to \footnotesize as the IEEE normally uses this smaller font for
% tables. The \label must come after \caption as always.
%
%\begin{table}[!t] % increase table row spacing, adjust to taste
%\renewcommand{\arraystretch}{1.3} if using array.sty, it might be a good idea
%to tweak the value of \extrarowheight as needed to properly center the text
%within the cells \caption{An Example of a Table} \label{table_example}
%\centering % Some packages, such as MDW tools, offer better commands for making
%tables % than the plain LaTeX2e tabular which is used here.
%\begin{tabular}{|c||c|} \hline One & Two\\
%\hline Three & Four\\
%\hline \end{tabular} \end{table}


% Note that the IEEE does not put floats in the very first column
% - or typically anywhere on the first page for that matter. Also, in-text
%   middle ("here") positioning is typically not used, but it is allowed and
%   encouraged for Computer Society conferences (but not Computer Society
%   journals). Most IEEE journals/conferences use top floats exclusively. Note
%   that, LaTeX2e, unlike IEEE journals/conferences, places footnotes above
%   bottom floats. This can be corrected via the \fnbelowfloat command of the
%   stfloats package.




\section{Conclusión}
 Luego de este análisis (y algunas cuentas), finalmente podemos responder
 nuestras conjeturas iniciales y quedarnos con nuestra consciencia limpia.

 Recopilando los resultados podemos concluir que:
 \begin{itemize}
   \item Ninguna de las poblaciones puede extinguirse siempre y cuando hayan
 conejos y zorros inicialmente, estas se quedarían oscilando indefinidamente
 siendo en ocasiones más prospero para una especie que para otra pero
 inevitablemente repitiendose el ciclo. 
   \item Si es que hubiese ausencia de alguna especie, la población de conejos
   creceria exponencialmente, y la de zorros desaparecería debido a la falta de
   alimento.
   \item Los métodos de estabilidad y aproximación numérica permitieron dar una
   clara perspectiva de nuestras incognitas sin resolverlas de manera de exacta.
   La diferencia es que en un caso obtuvimos un panorama del sistema para
   cualesquiera valores iniciales, cuando en contraparte, aproximar numéricamente
   nos requiere una en especifico. Este segundo método podría nos ser viable si
   lo que nos interesa es conocer una perspectiva del flujo para muchas
   condiciones iniciales.
   \item Si bien además al final se trabajaron sobre parámetros específicos en
   nuestro modelo, por lo que mostramos en la sección 2, el comportamiento
   oscilatorio alrededor de uno de los puntos fijos y el de punto silla en el
   origen de los ejes coordenados se cumple para cualquiera valores de
   $\alpha,\beta,\gamma,\delta$ positivos, variando la periodicidad oscilatoria
   y la velocidad de crecimiento en la población de tales especies.
 \end{itemize}
 




% if have a single appendix: \appendix[Proof of the Zonklar Equations] or
%\appendix  % for no appendix heading do not use \section anymore after
%\appendix, only \section* is possibly needed

% use appendices with more than one appendix then use \section to start each
% appendix you must declare a \section before using any \subsection or using
% \label (\appendices by itself starts a section numbered zero.)
%

% use section* for acknowledgment



% trigger a \newpage just before the given reference number - used to balance
% the columns on the last page adjust value as needed - may need to be
% readjusted if the document is modified later \IEEEtriggeratref{8} The
% "triggered" command can be changed if desired:
% \IEEEtriggercmd{\enlargethispage{-5in}}

% references section

% can use a bibliography generated by BibTeX as a .bbl file BibTeX documentation
% can be easily obtained at: http://mirror.ctan.org/biblio/bibtex/contrib/doc/
% The IEEEtran BibTeX style support page is at:
% http://www.michaelshell.org/tex/ieeetran/bibtex/ \bibliographystyle{IEEEtran}
% argument is your BibTeX string definitions and bibliography database(s)
% \bibliography{IEEEabrv,../bib/paper}
%
% <OR> manually copy in the resultant .bbl file set second argument of \begin to
% the number of references (used to reserve space for the reference number
% labels box)
\begin{thebibliography}{1}

\bibitem{IEEEhowto:strogartz}
S.~Strogartz, \emph{Non Linear Dynamics and Chaos}, \relax Reading,
Massachusetts: Perseus Books, 1994.

\end{thebibliography}

% biography section
%
% If you have an EPS/PDF photo (graphicx package needed) extra braces are needed
% around the contents of the optional argument to biography to prevent the LaTeX
% parser from getting confused when it sees the complicated \includegraphics
% command within an optional argument. (You could create your own custom macro
% containing the \includegraphics command to make things simpler here.)
% \begin{IEEEbiography}[{\includegraphics[width=1in,height=1.25in,clip,keepaspectratio]{mshell}}]{Michael
% Shell} or if you just want to reserve a space for a photo:

% You can push biographies down or up by placing a \vfill before or after them.
% The appropriate use of \vfill depends on what kind of text is on the last page
% and whether or not the columns are being equalized.

%\vfill

% Can be used to pull up biographies so that the bottom of the last one is flush
% with the other column. \enlargethispage{-5in}



% that's all folks
\end{document}


